\chapter{Konklusion}
\label{c:konklusion}

Igennem kapitlerne \ref{c:lexer}--\ref{c:kodegenerering} er Qjava sproget og Qjava compileren blevet konstrueret efter retningslinierne i kapitel \ref{c:compilerstruktur}. Resultatet er blevet en  velfungerende compiler, der genererer assemblerkode, der direkte kan assembleres og afvikles. Qjava compileren opfylder fuldt ud kravsspecifikationen givet i sektion \xref{s:kravsspecifikation} og undersektioner. 

Det eneste der kan synes problematisk er begr�nsningerne for \t{if, while} og metoders st�rrelse. Disse problemer skal dog ene og alene tilskrives assemblerspecifikke og Microsoft DOS specifikke problemer, og er ikke ``overs�tterteoretiske problemer''. I samme boldgade opstod der mindre problemer med hvilke registre der kunne udf�re hvilke funktioner.

I \kref{c:hastighedstest} blev tesen fra indledningen om at uoptimeret native assemblerkode er v�senligt hurtigere end afviklingen af bytecode falsificeret. Det viste sig, at for et lille, men  beregningsm�ssigt tungt, program var afviklingshastigheden kortere for JDK. JDK var 9 sekunder om at afvikle programmet, mens assemblerkoden Qjava compileren producerede, var 11 sekunder om samme job. Ved hj�lp af to simple optimeringsprincipper angivet, i kapitel \ref{c:hastighedstest}, blev k�retiden for assemblerkoden dog reduceret til 8 sekunder. Og yderligere 2 sekunder blev vundet ved gennemf�relse af mere vanskelig optimering. De henholdsvis 6 og 8 sekunder er hurtigere end JDK, men som tiderne viser, bliver tidsgevinsten ved anvendelse af native assembler nok aldrig en st�rrelsesorden eller i n�rheden af hvad der p� forh�nd var forventet.

Omvendt er dette budskab for Java programm�rer utrolig possitivt, da teser som den indledningen pr�senterede kan manes i jorden. Rapporten kunne m�ske ligefrem v�re med til at skabe en holdnings�ndring hos de mennesker der stadig mener Java er et langsomt sprog --- i hvertfald n�r diskusionen omhandler mindre programmer.


\begin{quote}\textit{Det m� derfor konkluderes, at anvendeligheden af Qjava compileren, set i et tidsbesparelses-perspektiv, ikke er specielt stor. Der skal implementere meget effektive optimeringsalgoritmer, hvis m�ls�tningen med Qjava compileren skal opfyldes.}
\end{quote}


Qjava compileren producerer assemblerkode med kommentarer, hvilket kan g�re compileren anvendelig for ikke tilsigtede m�lgrupper. Disse m�lgrupper kunne v�re kursister ved overs�tterteknik-kurser, eller kurser i Intel 8086 assembler, da det er muligt at f�lge Qjava programmernes gang igennem kommentarerne i assemblerkoden.


I forhold til modul 1 p� datalogi/RUC m�ls�tning om at l�re Java og forst� sprogets sammenh�nge, passer dette projekt godt, da det netop er sproget og ikke kodegenereringen, der er fokus. Projektet er endvidere et godt opl�g til et modul 2 projekt, der kunne udbygge compileren med  semantik kontrol, advanceret kodegenerering, lager-administration og kodeoptimering.

