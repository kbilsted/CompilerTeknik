\chapter{Parser}
\label{a:parser}
\OverviewLineNoTitle
\begin{footnotesize}\begin{verbatim}
import java.io.*;
import java.util.Vector;


public class Parser implements TokenNames
{
    private Lexer lexer;
    private boolean isEOF = false;
    private Token T;        // indeholder det aktuelle token


    // for constructing the SymbolTable (sbtb)
    private String sbtbFncname = null;
    private String sbtbClassname;
    private SymbolTable symbolTable;



// Constructor
Parser(Lexer lexer, SymbolTable symbolTable) throws IOException
{
    if(lexer != null)
    {
        this.lexer = lexer;
        T = lexer.getNextToken();   // l�s f�rste token
        this.symbolTable = symbolTable;
    }
    else
        errorAndExit("Lexer not available\n");
}


public Tree parse() throws IOException
{
    return( S() );// start parsing
}


private Tree S() throws IOException
{
    return( classdef () );
}


private Tree classdef() throws IOException
{
    Classdef tree = new Classdef(T.lineno);

    while(isEOF == false)
    {
        eat(CLASS);
        tree.add(T.sval);
        sbtbClassname = T.sval;

        eat(ID);                // verificer className er en <ID>
        eat(LBRACE);            // verificer "{"
        tree.add( classcontents() );
        eat(RBRACE);            // verificer "}"
    }

    return(tree);
}


private Tree classcontents() throws IOException
{
    Classcontents tree = new Classcontents(T.lineno);

    if(T.id != ID && T.id != VOID)
        errorAndExit("Error", "Expected <ID> or \"void\" got token #"+T.id, T.lineno);

    // vardef || fncdef
    while(T.id == ID || T.id == VOID || T.id == SEMICOLON)
    {
        if(T.id != SEMICOLON)
        {   // vardef
            if(T.id == ID)
                tree.add( vardef() );
            else
                 tree.add( fncdef() );
        }
        else
            eat(SEMICOLON);
    }

    return(tree);
}


private Tree vardef() throws IOException
{
    String type = T.sval;
    eat(ID);

    String name = T.sval;
    eat(ID);

    symbolTable.add(type, sbtbClassname, sbtbFncname, name);

    eat(SEMICOLON);
    return( new Vardef(type, name, T.lineno) );
}



private Tree fncdef() throws IOException
{
    Vector v = new Vector();

    eat(VOID);
    String name = T.sval;
    sbtbFncname = T.sval;
    eat(ID);
    eat(LPAR);

    // EBNF: [<ID> <ID> { "," <ID> <ID> } ]
    if(T.id == ID)
    {
        v.addElement(T.sval);
        eat(ID);

        v.addElement(T.sval);
        eat(ID);

        // EBNF: { "," <ID> <ID> }
        while(T.id == COMMA)
        {
            eat(COMMA);
            v.addElement(T.sval);
            eat(ID);
            v.addElement(T.sval);
            eat(ID);
        }
    }

    eat(RPAR);

    eat(LBRACE);
    Tree s = sentences();
    eat(RBRACE);

    sbtbFncname = null;
    return( new Fncdef(name, v, s, T.lineno) );
}




private Tree sentences() throws IOException
{
    Sentences tree = new Sentences(T.lineno);

    if(T.id != IF && T.id != WHILE && T.id != BREAK && T.id != RETURN && T.id != ID &&
       T.id != NAME && T.id != RBRACE && T.id != SEMICOLON)
        errorAndExit("Error", "Expecting either: \"if\", \"while\", \"break\", \"return\", 
                      <ID>, <NAME>, \";\", or \"}\"", T.lineno);

    while(T.id == IF || T.id == WHILE || T.id == BREAK || T.id == RETURN || 
          T.id == ID || T.id == NAME || T.id == SEMICOLON)
    {
        switch(T.id)
        {
            case SEMICOLON: eat(SEMICOLON); break;
            case IF:    tree.add(if_());    break;
            case WHILE: tree.add(while_()); break;
            case BREAK: tree.add(break_()); break;
            case RETURN:tree.add(return_());break;
            case ID:
            case NAME: int lookAhead = peek(); // read next char
                switch(lookAhead)
                {
                    case -1:   errorAndExit("Error", "Unexpected End Of File", T.lineno);
                    case ID:   tree.add(vardef()); break;
                    case LPAR: tree.add(fnccall());break;
                    case SET:  tree.add(assign()); break;
                    default: errorAndExit("Error", "Expected <ID>, \")\" or \"=\"", T.lineno);
                }
                break;

            default:
                errorAndExit("Error", "Something beyond my comprehension took place...!", T.lineno);
        }
    }

    return(tree);
}



private Tree fnccall() throws IOException
{
    String name = T.sval;

    Fnccall tree;

    if(T.id == ID)
    {
        eat(ID);
        tree = new Fnccall(name, true, T.lineno); // local call
    }
    else
    {
        eat(NAME);
        tree = new Fnccall(name, false, T.lineno); // not local call
    }

    eat(LPAR);

    // EBNF: [ <E> { , <E> } ]
    if(T.id != RPAR)
    {
        tree.add( E() );

        while(T.id == COMMA)
        {
            eat(COMMA);
            tree.add( E() );
        }
    }

    eat(RPAR);
    eat(SEMICOLON);

    return(tree);
}


private Tree if_() throws IOException
{
    eat(IF);
    eat(LPAR);
    Tree condCode = E();
    eat(RPAR);

    eat(LBRACE);
    Tree thenCode = sentences();
    eat(RBRACE);

    eat(ELSE);
    eat(LBRACE);
    Tree elseCode = sentences();
    eat(RBRACE);

    return( new If(condCode, thenCode, elseCode, T.lineno) );
}



private Tree while_() throws IOException
{
    eat(WHILE);
    eat(LPAR);
    Tree condCode = E();
    eat(RPAR);
    eat(LBRACE);
    Tree whileCode = sentences();
    eat(RBRACE);

    return( new While(condCode, whileCode, T.lineno) );
}


private Tree break_() throws IOException
{
    eat(BREAK);
    eat(SEMICOLON);

    return( new Break(T.lineno) );
}


private Tree return_() throws IOException
{
    eat(RETURN);

    if(T.id == LPAR) { eat(LPAR); eat(RPAR);}
    eat(SEMICOLON);

    return( new Return(T.lineno) );
}


private Tree assign() throws IOException
{
    int op;
    String name = T.sval;

    if(T.id == ID)
    {   op = ID;
        eat(ID);
    }
    else
    {   op = NAME;
        eat(NAME);
    }

    eat(SET);
    Tree E = E();
    eat(SEMICOLON);

    return( new Assign(name, op, E, T.lineno) );
}


private Tree E() throws IOException
{
    Tree t = E1();

    while(T.id == OR)
    {
        eat(OR);
        OpDual tree = new OpDual(OR, t, E1(),T.lineno);
        t = tree;
    }

    return(t);
}

private Tree E1() throws IOException
{
    Tree t = E2();

    while(T.id == AND)
    {
        eat(AND);
        OpDual tree = new OpDual(AND, t, E2(),T.lineno);
        t = tree;
    }
    return(t);
}


private Tree E2() throws IOException
{
    Tree t = E3();

    while(T.id == BITOR)
    {
        eat(BITOR);
        OpDual tree = new OpDual(BITOR, t, E3(), T.lineno);
        t = tree;
    }
    return(t);
}


private Tree E3() throws IOException
{
    Tree t = E4();
    while(T.id == BITAND)
    {
        eat(BITAND);
        OpDual tree = new OpDual(BITAND, t, E4(), T.lineno);
        t = tree;
    }
    return(t);
}


private Tree E4() throws IOException
{
    Tree t = E5();

    while(T.id == EQUAL || T.id == NEQUAL)
    {
        int op;

        if(T.id == EQUAL)
        {
            eat(EQUAL); op = EQUAL;
        }
        else // !=
        {
            eat(NEQUAL); op = NEQUAL;
        }
        OpDual tree = new OpDual(op, t, E5(), T.lineno);
        t = tree;
    }

    return(t);
}



private Tree E5() throws IOException
{
    Tree t = E6();

    while(T.id == LESS || T.id == LEQUAL)
    {
        int op;

        if(T.id == LESS)
        {
            eat(LESS); op = LESS;
        }
        else // <=
        {
            eat(LEQUAL); op = LEQUAL;
        }
        OpDual tree = new OpDual(op, t, E6(), T.lineno);
        t = tree;
    }

    return(t);
}


private Tree E6() throws IOException
{
    Tree t = E7();

    while(T.id == PLUS || T.id == MINUS)
    {
        int op;

        if(T.id == PLUS)
        {
            eat(PLUS); op = PLUS;
        }
        else // -
        {
            eat(MINUS); op = MINUS;
        }
        OpDual tree = new OpDual(op, t, E7(), T.lineno);
        t = tree;
    }
    return(t);
}




private Tree E7() throws IOException
{
    Tree t = E8();

    while(T.id == MULT || T.id == DIV || T.id == MODULO)
    {
        int op = 0; // = 0 to make Java happy

        if(T.id == MULT)
        {
            eat(MULT); op = MULT;
        }

        if(T.id == DIV)
        {
            eat(DIV); op = DIV;
        }

        if(T.id == MODULO)
        {
            eat(MODULO); op = MODULO;
        }
        OpDual tree = new OpDual(op, t, E8(), T.lineno);

        t = tree;
    }

    return(t);
}


private Tree E8() throws IOException
{
    if(T.id == NEW)
    {
        eat(NEW);
        OpCall tree = new OpCall(NEW, T.sval, T.lineno);
        eat(ID);
        eat(LPAR); eat(RPAR);

        return(tree);
    }
    else
    {
        return( E9() );
    }
}


private Tree E9() throws IOException
{
    if(T.id == NOT || T.id == MINUS)
    {
        int op;

        if(T.id == NOT)
        {
            eat(NOT); op = NOT;
        }
        else
        {
            eat(MINUS); op = MINUS;
        }
        return( new OpMonadic(op, E10(),T.lineno) );
    }
    else
    {
        return( E10() );
    }
}


private Tree E10() throws IOException
{
    if(T.id != ID && T.id != NAME)
    {
        return( E11() );
    }
    else
    {
        OpCall tree;

        if(T.id == ID)
        {
            tree = new OpCall(ID, T.sval, T.lineno);
            eat(ID);
        }
        else
        {
            tree = new OpCall(NAME, T.sval, T.lineno);
            eat(NAME);
        }

        if(T.id == LPAR)
        {
            eat(LPAR);

            tree.setIsFncCall(); // this is a fnc call not a variable

            if(T.id != 6)
            {
                tree.add( E() );

                while(T.id == COMMA)
                {
                    eat(COMMA);
                    tree.add( E() );
                }
            }

            eat(RPAR);
        }

        return(tree);
    }
}


private Tree E11() throws IOException
{
    String s;
    int n;
    switch(T.id)
    {
        case VAL_STRING:
            s = T.sval; eat(VAL_STRING);
            return( new OpConst(VAL_STRING, s, T.lineno) );

        case VAL_CHAR:
            s = T.sval; eat(VAL_CHAR);
            return( new OpConst(VAL_CHAR, s, T.lineno) );

        case VAL_INT:
            n= T.nval; eat(VAL_INT);
            return( new OpConst(VAL_INT, n, T.lineno) );

        case LPAR:
            eat(LPAR);
            Tree t = E();
            eat(RPAR);
            return(t);

        // null is for now just treated as a zero
        case NULL:
            eat(NULL);
            return( new OpConst(VAL_INT, 0, T.lineno) );


        default:
            errorAndExit("\nError", "Unexpected token: " + T.id + " (" + T.sval + " " + T.nval + ")", 
                         T.lineno);
    }
    return(null);// Only here because of Java<tm>
}



private int peek() throws IOException
{
    if(T.id == EOF)
        return(-1);

    Token tmp = lexer.getNextToken();
    lexer.pushBack();

    return(tmp.id);
}



private void eat(int id) throws IOException
{
    if(T.id != id)
        errorAndExit("Error", "Expected token #"+id+ " got #" + T.id, T.lineno);

    T = lexer.getNextToken();

    if(T.id == EOF)
        isEOF = true;
}


private void errorAndExit(String error)
{
    System.out.println(error);
    System.exit(0);
}


private void errorAndExit(String s, String s2, int l)
{
    System.out.println(s +"("+l+"): "+ s2);
    System.exit(0);
}


/* For testing only 

public static void main(String s[])
{
    try
    {   SymbolTable myst = new SymbolTable();
        Parser a = new Parser(new Lexer("d:\\modul1\\proj\\parsertester.txt"), myst);

        Tree root = a.parse();

        //gennneml�b af parsetr�et
        System.out.println("\n\ngennemgang\n----------\n");
        System.out.print(root.toString() );

        System.out.println("\nTest af symbolTable\n");
        System.out.println("A = " + myst.classSize("A"));
        System.out.println("B = " + myst.classSize("B"));

        System.out.println("B.c() = " + myst.fncSize("B","c"));
        System.out.println("B.c().x = " + myst.varIndex("B","c","x") );
        System.out.println("B.k = " + myst.varIndex("B", null, "k") );
        System.out.println("vartype i = " + myst.varType("A", "a", "i") );
        myst.add("char", "A", null, "c", 42);
        System.out.println("A.c = " + myst.varIndex("A", null, "c") );

    }
    catch(Exception e)
    {
        System.out.print("cought " +e);
    }
}
*/

} // EOC
\end{verbatim}\end{footnotesize}