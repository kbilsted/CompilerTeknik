\chapter{CodeGenerator}
\label{a:codegenerator}
\OverviewLineNoTitle
\begin{footnotesize}\begin{verbatim}
import java.util.Vector;

public class CodeGenerator implements TokenNames
{
    // user-specifik compileroptions
    private String HOBSIZE = "32605";

    // to generate unique labels
    private int X = 0;

    // temporary storage of program divided into the two assembler segments
    // and a block of startup code
    private AsmBlock segptr; // points to the AsmBlock to be written to
    private AsmBlock startcode=new AsmBlock(); // upstart code
    private AsmBlock stdfnc  = new AsmBlock(); // standard functions
    private AsmBlock dataseg = new AsmBlock(); // data segment code
    private AsmBlock codeseg = new AsmBlock(); // code segment code
    private AsmBlock maincode= new AsmBlock(); // code containing static void main()
    // other
    String className;   // name of class we are currently working in
    String fncName;     // name of fnc we are currently working in
    SymbolTable sbtb;   // symbol table containing all variables in parsetree


    // konstrukt�r
    CodeGenerator(SymbolTable sbtb){this.sbtb = sbtb;}


    private void errorAndExit(String s)
    {System.out.print("\n" + s); System.exit(0);}


    private void errorAndExit(String s, String s2, int l)
    {System.out.print(s + "("+l+"): " + s2); System.exit(0);}

    public String generate(Tree parsetree)
    {

        generateStartCode();
        segptr = codeseg; // write the whole thing to codeseg
        sGen((Classdef) parsetree);
        return( startcode.toString() + dataseg.toString() +
                stdfnc.toString()    + codeseg.toString() +
                maincode.toString()
              );
    }


    private void sGen(Classdef tree)
    {
        // a classdefinition is defined as two sepperate elements in the parsetree
        className = (String) tree.getNext();

        while(className != null)
        {
            fncName = null;
            classdefGen((Classcontents) tree.getNext() );

            className = (String) tree.getNext(); // read next elem
        }
    }


    private void classdefGen(Classcontents tree)
    {   classcontentsGen(tree); }


    private void classcontentsGen(Classcontents tree)
    {
        Tree t = tree.getNext();

        while(t != null)
        {
            if(t instanceof Vardef)
            {
                // do nothing, as all the variables are allocated on beforehand.
            }
            else // instanceof Fncdef
            {
                // special case if functionname == "main"
                if( ((Fncdef) t).getName().equals("main") )
                    fncdefMainGen((Fncdef) t);
                else
                    fncdefGen((Fncdef) t);
            }

            t = tree.getNext();
        }
    }



    private void vardefGen(Vardef tree)
    {
        return; // variables are handed by newGen() and fncdefGen()
    }



    private void fncdefMainGen(Fncdef tree)
    {
        int offset = 0;
        fncName = "main";
        String fnc_endlabel = className + "_" + fncName + "_end";

        segptr = maincode;

        segptr.add("\n\nSTART:");
        segptr.add("mov  ax, @data ; setup DOS program");
        segptr.add("mov  ds, ax");
        segptr.add("; allocate object main resides in");
        segptr.add("mov  cx, "+sbtb.classSize(className) );
        segptr.add("call new");
        segptr.add("mov  bp, sp     ; adjust bp to point at functions local variable");
        segptr.add("sub  bp, 2      ;");
        segptr.add("mov  bx, ax    ; store main's 'this' ");
        segptr.add("; execute mains contents");

        // put all fnc local variables on stack with value = 0
        int fncsize = sbtb.fncSize(className, fncName)/2;
        for(int i = 0; i < fncsize; i++)
            segptr.add("push 0    ; push local variables to stack ");

        // generate all the sentences in function
        sentencesGen((Sentences) tree.getSentence(), null, fnc_endlabel);
        segptr.add("jmp  SystemExit");
        segptr.add("END");

        fncName = null;
        segptr = codeseg; // write again to the codeseg segment
    }





    /*
     * all the argumentvariables have been put on stack - we must added them to the
     * SymbolTable in the fncCallGen() so their possition on the stack can be calculated
     */
    private void fncdefGen(Fncdef tree)
    {
        int offset = 0;
        fncName = tree.getName();
        Vector argList = tree.getArgList();

        String fnc_endlabel = className + "_" + fncName + "_end";

        segptr.add("\n\n"+className+"_" + fncName + " PROC" ); // declare PROC
        segptr.add("mov  bp, sp ; adjust bp to point at functions local variable");
        segptr.add("sub  bp, 2");

        // put all fnc local variables on stack with value = 0
        int fncsize = sbtb.fncSize(className, fncName);
        for(int i = 0; i < fncsize/2; i++)
            segptr.add("push 0 ; put all local variables on stack");


        // put arglist variable names in SymbolTable with their corresponding
        // index # on the stack.
        // index # = first local variable (2bytes) + return adr (2bytes)
        // the index is also negated, as argument-variables are fetched with [bp+x]
        // unlike local variables which are fetched with [bp-x]
        for(int index=0, i=0; i < argList.size(); i++)
        {
            sbtb.add((String) argList.elementAt(i++), className, fncName,
                     (String) argList.elementAt(i), -(4+index) );
            index += 2;
        }


        // generate all the sentences in function
        sentencesGen((Sentences) tree.getSentence(), null, fnc_endlabel);



        // fnc end_label must be set in front of the de-stacking of variables
        segptr.add(fnc_endlabel + ":");

        // de-stack local variables
        for(int i = 0; i < fncsize/2; i++)
            segptr.add("pop  ax   ; de-stack local variables");

        // end function
        segptr.add("ret");
        segptr.add(className+"_" + fncName + " ENDP" ); // declare PROC

        fncName = null;

        // clear SymbolTable of argument variables
        for(int i = 0; i < argList.size(); i++)
        {
            sbtb.remove((String) argList.elementAt(i++), className,
                     fncName, (String) argList.elementAt(i));
        }
    }



    // input <sentences>, endlabel == if/while or fnc if break is called outside,
    private void sentencesGen(Sentences tree, String endlabel, String fnc_endlabel)
    {
        Tree t = tree.getNext();
        while(t != null)
        {
            if(t instanceof Vardef)
                vardefGen((Vardef) t);
            else
            if(t instanceof Fnccall)
                fnccallGen((Fnccall) t);
            else
            if(t instanceof If)
                ifGen((If) t, fnc_endlabel);
            else
            if(t instanceof While)
                whileGen((While) t, fnc_endlabel);
            else
            if(t instanceof Break)
                breakGen((Break) t, endlabel);
            else
            if(t instanceof Return)
                segptr.add("jmp  "+fnc_endlabel+" ; return");
            else
            if(t instanceof Assign)
                assignGen((Assign) t);

            t = tree.getNext();
        }
    }



    private void fnccallGen(Fnccall tree)
    {
        String name = tree.getName();

        segptr.add("; fnccallGen");

        // standard functions
        if(tree.getName().equals("System.out.print") )
        {
            segptr.add("push bp");
            segptr.add("push bx");
            eGen((Tree)tree.getArgList().elementAt(0));
            segptr.add("call SystemOutPrint");
            segptr.add("pop  cx ; de-stack argument");
            segptr.add("pop  bx");
            segptr.add("pop  bp");
            return;
        }

        if(tree.getName().equals("System.in.read"))
        {
            segptr.add("call SystemInRead");
            segptr.add("push ax ; store read letter");
            return;
        }

        if(tree.getName().equals("System.exit"))
        {
            segptr.add("jmp  SystemExit");
            return;
        }

        if(tree.getName().equals("Integer.toString"))
        {
            // generate first argument and put result in ax
            segptr.add("push bp");
            segptr.add("push bx");
            eGen((Tree)tree.getArgList().elementAt(0));
            segptr.add("call Integer_toString");
            segptr.add("pop  cx ; de-stack arguments");
            segptr.add("pop  bx");
            segptr.add("pop  bp");
            segptr.add("push ax ; store generated String");
            return;
        }


        // else non-local non standard function

        String callFnc   ="";// used in if isCallLocal == false
        String callClass ="";// used in if isCallLocal == false

        if(tree.isCallLocal() == false)
        {
            // we must first extract Obj pointer name  == classptr
            int sepp = name.indexOf('.');
            String classptr = name.substring(0, sepp);
            callFnc = name.substring(sepp+1);

            // find the pointers type to determine which class to call
            callClass = sbtb.varType(className, fncName, classptr);

            // reference variable is defined in class scope
            if(callClass == null)
            {
                callClass = sbtb.varType(className, null, classptr);
                int stackPos = sbtb.varIndex(className, null, classptr);

                segptr.add("push bp");
                segptr.add("push bx");
                segptr.add("mov  bx, hob[" + stackPos + "] ; push callers class' 'this'");
            }
            else
            {
                // fetch 'this' for the class that the pointer points at
                int stackPos = sbtb.varIndex(className, fncName, classptr);

                if(stackPos != -1)
                {
                    segptr.add("push bp");
                    segptr.add("push bx");
                    segptr.add("mov  bx, [bp-" + stackPos + "] ; push callers class' 'this'");
                }
                else
                    errorAndExit("Error", "Functioncall '"+name+"()' is unknown!", tree.lineno());
            }
        }
        else // local non standard function
        {
            segptr.add("push bp");
            // since the call is local, 'this' (bx) is set up correctly
        }


       /* push all arguments to stack (local variables will be pushed as the
        * first thing in the funktion)
        * The names will later be connected with the stack possition (in the
        * begining of the function, but generated in fncdefGen() )
        * All we do here is putting the values on the stack in reverse order, so
        * the varIndex in SymbolTable works correctly
        */
        Vector v = tree.getArgList();
        for(int i = v.size()-1; i >= 0 ; i--)
            eGen((Tree) v.elementAt(i));

        if(tree.isCallLocal() == true)
            segptr.add("call "+className+"_"+name);
        else
        {
            segptr.add("call " + callClass +"_"+ callFnc);
        }


        // since functions normaly doesn't support return-value we
        // have to do something special in the cases where we need a
        // returnvalue
        for(int i = 0; i < v.size(); i++)
            segptr.add("pop  cx ; de-stack arguments");

        if(tree.isCallLocal() == false)
        {
            segptr.add("pop  bx ; restore this class' 'this'");
            segptr.add("pop  bp");
        }
        else // only restore bp in local calls
        {
            segptr.add("pop bp");
        }

        // we do not have returnvalues for functions, but in these special
        // cases we need it and so "emulate" it.
        if(callFnc.equals("length") ||
           callFnc.equals("charAt") ||
           callFnc.equals("concat")
          )
            segptr.add("push  ax ; 'emulated' return value on the stack");

    }



    private void ifGen(If tree, String fnc_endlabel)
    {
        int x = X++;
        final String ifend_label = "endif_"+X;
        final String else_label  = "else_"+X;
        final String then_label  = "then_"+X;

        // gen IF
        eGen(tree.getCond());
        segptr.add("pop  cx     ; start if_"+x);
        segptr.add("cmp  cx, 0");
        segptr.add("je   " + else_label); // if CX == 0

        // gen THEN
        segptr.add(then_label+":");
        sentencesGen((Sentences) tree.getThen(), null, fnc_endlabel);
        segptr.add("jmp  " + ifend_label);

        // gen ELSE
        segptr.add(else_label+":");
        sentencesGen((Sentences) tree.getElse(), null, fnc_endlabel);
        segptr.add(ifend_label+":");
    }




    private void whileGen(While tree, String fnc_endlabel)
    {
        X++;
        String while_startlabel = "start_while"+X;
        String while_endlabel = "end_while"+X;

        // condition
        segptr.add(while_startlabel+":");
        segptr.add("; condition code");
        eGen(tree.getCond() );

        segptr.add("pop  ax ; compare condition code");
        segptr.add("cmp  ax, 0");
        segptr.add("je   " + while_endlabel + " ; condition is false");
        sentencesGen((Sentences) tree.getWhile(), while_endlabel, fnc_endlabel);
        segptr.add("jmp  " + while_startlabel + " ; loop once more");
        segptr.add(while_endlabel+":");
    }


    private void breakGen(Break tree, String endlabel)
    {
        if(endlabel == null)
        {
            String fnc;
            if(fncName != null)
                fnc = fncName + "() ";
            else
                fnc = " ";

            errorAndExit("Error", "\"break\" is only allowed inside while-scopes!", tree.lineno());
        }
        else
            segptr.add("jmp  "+endlabel+" ; break");
    }



    private void assignGen(Assign tree)
    {   segptr.add("; assign");

        eGen(tree.getE());
        segptr.add("pop  ax ; get value (assign)");

        if(tree.getOp() == ID)
        {
            String varname = tree.getName();
            int index = sbtb.varIndex(className, null, varname);

            if(index >= 0)
            {
                segptr.add("mov  hob[bx+"+index+"], ax ; set class variable " + varname);
            }
            else// not a class variable, but a fnc variable
            {
                index = sbtb.varIndex(className, fncName, varname);

                if(index == -1)
                    errorAndExit("Error","Variable "+varname+" not defined in current scope.", 
                                  tree.lineno());

                // if variable is from arguments, they must be fetched differently
                if(index < 0)
                {
                    segptr.add("mov  [bp+"+(index*-1)+"], ax ; set argument variable " + varname);
                }
                else
                    segptr.add("mov  [bp-"+index+"], ax ; set local variable " + varname);
            }
        }
        else // getOp() == NAME
        {
            String name = tree.getName();

            // we must first extract Obj pointer name
            int sepp = name.indexOf('.');
            String classptr = name.substring(0, sepp);
            String varname = name.substring(sepp+1);

            // find the class ptr is pointing at
            String callClass = sbtb.varType(className, fncName, classptr);

            int classptrIndex = sbtb.varIndex(className, null, varname);
            int varindexOtherClass = sbtb.varIndex(callClass, null, varname);

            if(varindexOtherClass >= 0)
            {
                // get value p points at (in p.a)
                segptr.add("push bx");
                segptr.add("                            ; value of " + name);
                segptr.add("mov  bx, [bp-"+(classptrIndex)+"] ; this of other class");
                // get value of a (in p.a)
                segptr.add("mov  hob[bx+"+varindexOtherClass+"], ax ; set value of variable 
                                                                      in other class");
                segptr.add("pop  bx");
            }
            else
            {
                errorAndExit("Error", "Variable \""+name+"\" does not exist.", tree.lineno());
            }
        }
    }


    // generate <E>
    private void eGen(Tree e)
    {
        if(e instanceof OpMonadic)  opMonadicGen((OpMonadic) e);
        else
        if(e instanceof OpDual)     opDualGen((OpDual) e);
        else
        if(e instanceof OpConst)    opConstGen((OpConst) e);
        else
        if(e instanceof OpCall)     opCallGen((OpCall) e);
        else
            errorAndExit("Internal error: Tried to generate <E> - but tree was no <E>!");
    }


    private void opMonadicGen(OpMonadic tree)
    {   segptr.add("; monadic begin");
        eGen(tree.getR() );
        segptr.add("pop  ax");

        switch(tree.getOp())
        {
            case MINUS:
                segptr.add("neg  ax");
                segptr.add("push ax");
                break;

            case NOT:
                segptr.add("not  ax");
                segptr.add("push ax");
                break;
        }
    }



    private void opDualGen(OpDual tree)
    {
        eGen(tree.getL() );
        segptr.add("; ");
        eGen(tree.getR() );

        switch(tree.getOp())
        {
            case OR:
            case BITOR:
                segptr.add("pop  dx");
                segptr.add("pop  ax");
                segptr.add("or   ax, dx");
                segptr.add("push ax");
                break;

            case AND:
            case BITAND:
                segptr.add("pop  dx");
                segptr.add("pop  ax");
                segptr.add("and  ax, dx");
                segptr.add("push ax");
                break;

            case EQUAL:
                X++;
                segptr.add("; equal");
                segptr.add("pop  dx");
                segptr.add("pop  ax");
                segptr.add("cmp  ax, dx");
                segptr.add("je   eq_"+X);
                segptr.add("xor  ax, ax  ; is !="); // false result
                segptr.add("jmp  eq_end"+X);
                segptr.add("eq_"+X+":");
                segptr.add("mov  ax, 65535 ; is =="); // true result
                segptr.add("eq_end"+X+":");
                segptr.add("push ax");
                break;

            case NEQUAL:
                X++;
                segptr.add("pop  dx");
                segptr.add("pop  ax");
                segptr.add("cmp  ax, dx");
                segptr.add("jne  neq_"+X); // remember 2 != 3 is true
                segptr.add("xor  ax, ax  ; is ==");
                segptr.add("jmp  neq_end"+X);
                segptr.add("neq_"+X+":");
                segptr.add("mov  ax, 65535 ; is !="); // true result
                segptr.add("neq_end"+X+":");
                segptr.add("push ax");
                break;

            case LESS:
                X++;
                segptr.add("pop  dx");
                segptr.add("pop  ax");
                segptr.add("cmp  ax, dx");
                segptr.add("jl   less_"+X);
                segptr.add("xor  ax, ax ; is not <");
                segptr.add("jmp  less_end"+X);
                segptr.add("less_"+X+":");
                segptr.add("mov  ax, 65535 ; is <"); // true result
                segptr.add("less_end"+X+":");
                segptr.add("push ax");
                break;

            case LEQUAL:
                X++;
                segptr.add("pop  dx");
                segptr.add("pop  ax");
                segptr.add("cmp  ax, dx");
                segptr.add("jle  lequal_"+X);
                segptr.add("xor  ax, ax  ; is not <=");
                segptr.add("jmp  lequal_end"+X);
                segptr.add("lequal_"+X+":");
                segptr.add("mov  ax, 65535 ; is <="); // true result
                segptr.add("lequal_end"+X+":");
                segptr.add("push ax");
                break;

            case PLUS:
                segptr.add("pop  dx");
                segptr.add("pop  ax");
                segptr.add("add  ax, dx");
                segptr.add("push ax");
                break;

            case MINUS:
                segptr.add("pop  dx");
                segptr.add("pop  ax");
                segptr.add("sub  ax, dx");
                segptr.add("push ax");
                break;


            case MULT:
                segptr.add("pop  dx");
                segptr.add("pop  ax");
                segptr.add("mul  dx");
                segptr.add("push ax");
                break;


            case DIV:
                segptr.add("pop  si");
                segptr.add("pop  ax");
                segptr.add("xor  dx, dx");
                segptr.add("div  si");
                segptr.add("push ax");
                break;

            case MODULO:
                segptr.add("pop  si");
                segptr.add("pop  ax");
                segptr.add("xor  dx, dx");
                segptr.add("div  si");
                segptr.add("push dx");
                break;
        }
    }

    private void opConstGen(OpConst t)
    {
        switch(t.getOp())
        {
            case VAL_INT:
                segptr.add("mov  ax, " + t.getNval() );
                segptr.add("push ax");
                break;

            case VAL_CHAR:
                segptr.add("mov  ax, '" + t.getSval() + "'");
                segptr.add("push ax");
                break;

            case VAL_STRING:
                String s = t.getSval();
                int len = s.length();
                segptr.add("; string - allocate class and return pointer on stack");
                segptr.add("mov  cx, "+((2*len)+2) + " ; length of 2*str + 2 ");
                segptr.add("call new");
                segptr.add("mov  si, ax ; mov (adr of obj in hob) to si");

                segptr.add("mov  hob[si], "+len+" ; insert strlength");
                for(int i = 0, index = 2; i < len; i++, index+=2)
                    segptr.add("mov  hob[si+"+index+"], '"+ s.charAt(i)+ "'");

                segptr.add("push ax");
                break;

            default: errorAndExit("Internal error in opConstGen()");
        }
    }



    private void opCallGen(OpCall tree)
    {
        // variable from this object
        if(tree.getOp() == ID && tree.isFncCall() == false)
        {
            String varname = tree.getName();
            int index = sbtb.varIndex(className, fncName, varname);

            // variable is declared in class-scope
            if(index == -1)
            {
                index = sbtb.varIndex(className, null, varname);

                // if still unknown the variable does not exist
                if(index == -1)
                    errorAndExit("Error", "variable \"" + tree.getName() +
                                  "\" is not defined within the scope it was used.", tree.lineno());

                segptr.add("mov  ax, hob[bx+"+index+"] ; get class variable " + varname);
                segptr.add("push ax");
                return;
            }
            if(index < 0)
            {
                segptr.add("mov  ax, [bp+"+(index*-1)+"] ; get argument variable " + varname);
                segptr.add("push ax");
            }
            else
            {
                segptr.add("mov  ax, [bp-"+(index)+"] ; get local variable " + varname);
                segptr.add("push ax");
            }
        }
        else
        // variable from other object
        if(tree.getOp() == NAME && tree.isFncCall() == false)
        {

            // we must first extract Obj pointer name
            int sepp = tree.getName().indexOf('.');
            String classptr = tree.getName().substring(0, sepp); // name of class, "p" in p.a
            String varname = tree.getName().substring(sepp+1);   // name of var,   "a" in p.a

            // find the class ptr is pointing at
            String callClass = sbtb.varType(className, fncName, classptr);

            if(callClass != null)
            {
                int classptrIndex = sbtb.varIndex(className, null, varname);
                int varindexOtherClass = sbtb.varIndex(callClass, null, varname);

                // if still unknown the variable does not exist
                if(varindexOtherClass == -1)
                    errorAndExit("Error", "variable \""+varname+"\" is not defined in class "
                                  +callClass, tree.lineno());

                // get value p points at (in p.a)
                segptr.add("mov  ax, [bp-"+(classptrIndex)+"] ; this of other class");
                // get value of a (in p.a)
                segptr.add("mov  ax, hob[ax+"+varindexOtherClass+"] ; get value of variable  
                                                                      in other class");
                segptr.add("push ax");
            }
            else
                errorAndExit("Error", "variable \""+classptr+"\" is not defined within the 
                                                                 scope it is being used.",
                             tree.lineno());
        }
        else
        if(tree.getOp() == NEW)
        {
            segptr.add("; new");
            segptr.add("mov  cx, "+ sbtb.classSize(tree.getName()) + " ; size of class " + 
                        tree.getName() );
            segptr.add("call new");
            segptr.add("push ax ; save adr pointer of object");
        }
        else // fnccall with <ID>/<NAME>
        {
            Fnccall tmp;

            if(tree.getOp() == NAME)
                tmp = new Fnccall(tree.getName(), tree.getArgList(), false);
            else
                tmp = new Fnccall(tree.getName(), tree.getArgList(), true);

            fnccallGen(tmp);
        }
    }

    // All the code necesary for a program
    private void generateStartCode()
    {
        startcode.add("DOSSEG");
        startcode.add(".MODEL SMALL");
        startcode.add(".STACK 100h");

        dataseg.add("\n\n.DATA");
        dataseg.add("hobptr  dw 0");
        dataseg.add("hob     dw "+HOBSIZE+" dup(0) ; allocate hob");
        dataseg.add("; error messages");
        dataseg.add("oomstr db  10,13,'Out of memory! Can not allocate another class.','$',0,0");


        stdfnc.add("\n\n.CODE");
        stdfnc.add("jmp  START\n");

        stdfnc.add("; ------------ STANDARD FUNCTIONS BEGIN ----------:");
        stdfnc.add("\nnew PROC");
        stdfnc.add("mov  ax, hobptr     ; get hop size");
        stdfnc.add("push ax            ; save pos. of class in stack");
        stdfnc.add("add  ax, cx        ; add size of obj");
        stdfnc.add("cmp  ax, "+HOBSIZE);
        stdfnc.add("jg   OutOfMem        ; if ax < HOBSIZE jump to OutOfMemory");
        stdfnc.add("mov  hobptr, ax     ; save hop size");
        stdfnc.add("pop  ax             ;   ax = pos. of class in hob");
        stdfnc.add("ret");
        stdfnc.add("OutOfMem:");
        stdfnc.add("mov  dx, OFFSET oomstr; write out of mem string to screen");
        stdfnc.add("mov  ah, 9h");
        stdfnc.add("int  21h");
        stdfnc.add("jmp  SystemExit");
        stdfnc.add("new  ENDP");


        // prints String's only!
        stdfnc.add("\nSystemOutPrint PROC");
        stdfnc.add("mov  bp, sp");
        stdfnc.add("sub  bp, 2");
        stdfnc.add("mov  bx, [bp+4]     ; get adr of str obj");
        stdfnc.add("mov  cx, hob[bx]    ; strlen");
        stdfnc.add("mov  si, bx         ; source-print ptr");
        stdfnc.add("add  si, 2          ; get past length field");
        stdfnc.add("printloop:");
        stdfnc.add("mov  dx, hob[si]");
        stdfnc.add("add  si, 2          ; next char");
        stdfnc.add("mov  ah, 2h         ; print char");
        stdfnc.add("int  21h");
        stdfnc.add("loop printloop");
        stdfnc.add("ret");
        stdfnc.add("SystemOutPrint ENDP");


        stdfnc.add("\nSystemInRead PROC");
        stdfnc.add("mov  ah, 1h ; read with screen echo");
        stdfnc.add("int  21h");
        stdfnc.add("mov  ah, 0 ; clear AH as only AL contains userinput");
        stdfnc.add("ret");
        stdfnc.add("SystemInRead ENDP");


        stdfnc.add("\nSystemExit PROC");
        stdfnc.add("mov  ah, 4ch");
        stdfnc.add("int  21h");
        stdfnc.add("ret");
        stdfnc.add("SystemExit ENDP");


        stdfnc.add("\nInteger_toString PROC");
        stdfnc.add("mov  bp, sp");
        stdfnc.add("sub  bp, 2");
        stdfnc.add("mov  ax, [bp+4]    ; get number (first argument)");
        stdfnc.add("xor  cx, cx        ; cx counts size of str");
        stdfnc.add("xor  di, di        ; flag for negative numbers");
        stdfnc.add("cmp  ax, 0         ; is negative?");
        stdfnc.add("jge  nonneg");
        stdfnc.add("inc  di            ; di == 1 == number is negative");
        stdfnc.add("inc  cx");
        stdfnc.add("neg  ax            ; make the number positive");
        stdfnc.add("nonneg:");
        stdfnc.add("mov  si, 10        ; div with reg SI as many times as possible");
        stdfnc.add("getDigits:");
        stdfnc.add("xor  dx, dx");
        stdfnc.add("div  si");
        stdfnc.add("add  dx, 48        ; conv digit to ASCII");
        stdfnc.add("push dx");
        stdfnc.add("inc  cx");
        stdfnc.add("cmp  ax, 0         ; are we done?");
        stdfnc.add("jg   getDigits\n");
        stdfnc.add("push cx            ; store cx");
        stdfnc.add("shl  cx, 1         ; calc str size = (2*strlen)+2");
        stdfnc.add("add  cx, 2");
        stdfnc.add("call new           ; alloc new str");
        stdfnc.add("pop  cx            ; get original strlen");
        stdfnc.add("mov  si, ax        ; ptr to str in hob");
        stdfnc.add("mov  hob[si], cx   ; store size of str");
        stdfnc.add("cmp  di, 0         ; is no negative?");
        stdfnc.add("je   toStr");
        stdfnc.add("add  si, 2         ; next pos");
        stdfnc.add("mov  hob[si], '-'  ; write '-' sign");
        stdfnc.add("dec  cx");
        stdfnc.add("toStr:");
        stdfnc.add("add  si, 2         ; next pos");
        stdfnc.add("pop  dx            ; get digit");
        stdfnc.add("mov  hob[si], dx   ; store size of str");
        stdfnc.add("loop toStr");
        stdfnc.add("ret");
        stdfnc.add("Integer_toString ENDP");




        stdfnc.add("\nString_charAt PROC");
        stdfnc.add("mov  bp, sp");
        stdfnc.add("sub  bp, 2");
        stdfnc.add("mov  si, bx        ; pos in hob where obj begins");
        stdfnc.add("mov  ax, [bp+4]    ; get (first) argument telling pos to get");
        stdfnc.add("shl  ax, 1         ; ax = ax * 2 every char is 2 elems");
        stdfnc.add("add  si, ax        ; pos in hob to fetch character");
        stdfnc.add("mov  ax, hob[si+2] ; +2 as first field contains str_len");
        stdfnc.add("ret");
        stdfnc.add("String_charAt ENDP");


        // bx is source ptr
        stdfnc.add("\nString_concat PROC");
        stdfnc.add("mov  bp, sp");
        stdfnc.add("sub  bp, 2");
        stdfnc.add("mov  cx, hob[bx] ; size of caller str");
        stdfnc.add("mov  si, [bp+4]  ; pos in hob of 2nd str len");
        stdfnc.add("add  cx, hob[si] ; add 2nd str len");
        stdfnc.add("push cx");
        stdfnc.add("add  cx, 2      ; add space for sizefield");
        stdfnc.add("call new        ; alloc new String object");
        stdfnc.add("pop  cx         ; total strsize");
        stdfnc.add("mov  si, ax      ; can't just use the AX");
        stdfnc.add("mov  hob[si], cx ; set concat'ed String size\n");
        stdfnc.add("mov  cx, hob[bx] ; size of str1");
        stdfnc.add("add  bx, 4       ; start in 2nd pos (but why +4 instead of +2 ??)");
        stdfnc.add("mov  di, ax      ; destination ptr = adr of new String obj");
        stdfnc.add("add  di, 2       ; start in 2nd pos");
        stdfnc.add("strcat1:");
        stdfnc.add("mov  si, [bx]    ; we can't just use [bx]");
        stdfnc.add("mov  hob[di], si");
        stdfnc.add("add  bx, 2");
        stdfnc.add("add  di, 2");
        stdfnc.add("loop strcat1\n");
        stdfnc.add("mov  si, [bp+4]  ; pos in hob of 2nd str len");
        stdfnc.add("mov  cx, hob[si] ; size of 2nd str len");
        stdfnc.add("mov  bx, [bp+4]  ; point at start of 2nd string");
        stdfnc.add("add  bx, 4       ; start in 2nd pos (but why +4??)");
        stdfnc.add("strcat2:");
        stdfnc.add("mov  si, [bx]    ; we can't just use [bx]");
        stdfnc.add("mov  hob[di], si");
        stdfnc.add("add  bx, 2");
        stdfnc.add("add  di, 2");
        stdfnc.add("loop strcat2");
        stdfnc.add("ret");
        stdfnc.add("String_concat ENDP");


        stdfnc.add("\nString_length PROC ; str_len length()");
        stdfnc.add("mov  ax, hob[bx] ; first field in string");
        stdfnc.add("ret");
        stdfnc.add("String_length ENDP");

        stdfnc.add("; ------------ STANDARD FUNCTIONS END ----------:");
    }
}
\end{verbatim}\end{footnotesize}
