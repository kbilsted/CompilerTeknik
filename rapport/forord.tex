\newpage
\pagenumbering{Roman}
\chapter*{Forord}
Med udgivelsen af denne nye version af projektrapporten, h�ber jeg at de fleste fejl der havde sneget sig ind i afleveringsversionen er elimineret. Endvidere har erfaringer fra den mundlige eksamination medf�rt to nye appendix. Det har dog v�ret mit m�l med denne nye version, at lade rapporten fremst� som jeg gerne ville have haft afleveret den, fremfor at omrevidere den. 

I forbindelse med den mundlige eksamen (i januar 2000), oversatte jeg primtalsprogrammet til C og compilerede programmet i Microsofts Visual C++ '97 som gav en exe-fil med en k�retid p� 8 sekunder. Herefter programmerede jeg en assemblerversion ``som en assembler programm�r ville have skrevet programmet'' (koden findes i \aref{a:asmtestprogram}). Programmet havde en k�retid p� 5 sekunder. Set i dette lys er Qjava compilerens performance (medtaget de to sm� foresl�ede optimeringer), ikke s� ringe som antydet i rapporten.

Versionshistorie:

\begin{description}
\item[Version 1.0, 23 dec. 1999] Den oprindelige version afleveret til eksamination.

\item[Version 1.01, 4 feb. 2000] Der er rettet stave og kommafejl, samt en mindre redigering af ford�kte s�tningskonstruktioner. Endvidere er henvisninger korrigeret og figurer justeret. Endelig er appendix \ref{a:qjavaoutput}, \ref{a:asmtestprogram} tilf�jet, der viser udsnit af Qjava compilerens kodegenerering (af primtalstest programmet) og en h�ndskrevet assembler version af samme program.
\end{description}

Rapporten er trods sin l�ngde og omfang blevet udf�rdiget p� normeret tid, dvs. 1/2 semester. For dem der kunne have interesse, kan det til slut n�vnes, at rapporten blev bed�mt til 13. 

\medskip

Jeg h�ber du vil f� gl�de og inspiration ved l�sning af rapporten\\
--Kasper B. Graversen, feb. 2000
