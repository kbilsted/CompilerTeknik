%
% Forsiden + abstract
%

\begin{titlepage}
\def\streg{ \rule{\textwidth}{0.4mm}}
~\vskip 0.6cm

 \Large{\textbf{Datalogi, modul 1\\Roskilde Universitetcenter\\Vejleder: Mads Rosendahl}}
\vskip 2.4cm

\streg
\vskip 3mm

\begin{Huge}
\hskip 4.27cm Compilerteknik,\vskip 1pt
\hskip 4.27cm fra Java til Assembler\vskip 5pt
\end{Huge}

\begin{Large}
\hskip 4.27cm  Efter�r 1999\vskip 1pt
\end{Large}
\begin{small}
\hskip 4.27cm  version 1.01
\end{small}
\vskip 0.05 cm
\streg

\vskip 5.8cm
\begin{Large}
\textbf{Forfattet af:}\\
\textit{Kasper B. Graversen}\\
\end{Large}
{\small vha \LaTeX2e\ }

\end{titlepage}

\begin{abstract}
Projektet udgangspunkt var f�lgende tese:

\textit{Ved at anvende en compiler, der kan overs�tte et Java program til native kode, kan programmer afvikles hurtigere end JDK's virtual machine. Dette g�lder b�de med hensyn til opstart, og den faktiske udf�rsel ogs� selvom den producerede kode ikke optimeres.}

\textit{Compileren, dette projekt pr�senterer, kan derfor anvendes til compilering af sm� nyttige programmer, der herved opn�r at v�re mindre ressourcekr�vende og hurtigere i deres udf�rsel.}

Udfra dette skabes et programmeringssprog, Qjava, der er en lille delm�ngde af Java. Endvidere realiseres en Qjava compiler i programmeringsproget Java, der kunne overs�tte Qjava kode til assembler, der efterf�lgende kunne assembleres og afvikles p� en DOS maskine. 

Igennem st�rk afgr�nsning af sprog, fejlkontrol mv. lykkedes det at skabe en velfungerende compiler. Id�en bag compileren er, at brugerens programmer skal skabes og kunne fungere i JDK. Qjava compileren skal f�rst anvendes, n�r brugeren �nsker en hurtigere afvikling af sit f�rdige program. S�ledes skal javaprogrammer, der er indeholdt i Qjava, og som kan compileres i JDK ogs� kunne compileres i Qjava compileren.

Efter realiseringen af compileren blev et regnetungt primtalsprogram testet i henholdsvis JDK og i Qjava. Resultatet var, at JDK afviklede koden hurtigere, end den uoptimerede assembler Qjava compileren kunne generere! Tesen, der var igangs�tter af projektet, var falsificeret. 

Rapporten konkluderer herefter:

\textit{Anvendeligheden af Qjava compileren, set i et tidsbesparelses-perspektiv, ikke er specielt stor. Der skal implementeres meget effektive optimeringsalgoritmer, hvis m�ls�tningen med Qjava compileren skal opfyldes.}


\end{abstract}
