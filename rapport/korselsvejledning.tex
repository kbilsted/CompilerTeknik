\chapter{K�rselsvejledning}
For at skabe en \t{*}.exe fil, der kan afvikles under DOS, skal man igennem flere stadier, denne korte vejledning vil vise hvilke og hvordan. Vejledningen tager udgangspunkt i et korrekt installeret Javamilj� (JDK) samt assemblermilj� (TASM).

\OverviewLineNoTitle

\subsubsection{.java $\longrightarrow$ .asm}
 F�rst skal  ens Java sourcekode compileres til assemblerkode, dette g�res ved at skrive

\verb�C:\>java Qjava filnavn�

 Dette afvikler Qjava compileren, der l�ser sourcekoden ``filnavn.java'', og skriver filen ``filnavn.asm''. Bem�rk, at der ingen endelse findes p� filen der gives som argument til compileren.

\subsubsection{.asm $\longrightarrow$ .exe}
Sidste stadie kr�ver b�de en assemblering og en linkning. Dette udf�res ved at skrive

\verb�C:\>tasm filnavn.asm�

\verb�C:\>tlink filnavn.obj�

Hvis alt har forl�bet problemfrit, findes nu ``filnavn.exe'', der kan afvikles.


