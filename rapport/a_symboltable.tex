\chapter{SymbolTable}
\label{a:symboltable}
\OverviewLineNoTitle
\begin{footnotesize}\begin{verbatim}
import java.util.Vector;

public class SymbolTable
{
    private Vector list = new Vector(); // contains elements

    protected class Symbol
    {
        int index;
        String fromClass, fromFnc, name, type;

        // konstrukt�r
        Symbol(String t, String fC, String fF, String n, int i)
        {type = t; fromClass = fC; fromFnc = fF; name = n; index = i;}
    }

    
    
    public void add(String type, String fromClass, String fromFnc, String name, int index)
    {
        this.add(type, fromClass, fromFnc, name);
        
        Symbol s = (Symbol) list.lastElement();
        s.index = index;
        
        list.setElementAt(s, list.size()-1); // overwrite element with the new index value
    }       

        
    public void add(String type, String fromClass, String fromFnc, String name)
    {
        Symbol s;
        int size = 0; // the index possition
        
        if(fromFnc == null)
        {
            // first find no. variables in same scope
            for(int i = 0; i < list.size(); i++)
            {
                s = (Symbol) list.elementAt(i);
                // same class but no function-scope
                if(fromClass.equals(s.fromClass) && fromFnc == null) 
                    size++;
            }
        }
        else
        {
            // first find no. variables in same scope
            for(int i = 0; i < list.size(); i++)
            {
                s = (Symbol) list.elementAt(i);
                // same class but no function-scope
                if(fromClass.equals(s.fromClass) && fromFnc.equals(s.fromFnc) ) 
                    size++;
            }
        }
                
        s = new Symbol(type, fromClass, fromFnc, name, size*2);

        list.addElement(s);
    }


    public void remove(String type, String fromClass, String fromFnc, String name)
    {
        Symbol s;
        
        for(int i = 0; i < list.size(); i++)
        {
            s = (Symbol) list.elementAt(i);
            
            // we must handle null-pointers sepperatly
            if(fromFnc == null)
            {
                if(fromClass.equals(s.fromClass) && s.fromFnc == null &&
                   name.equals(s.name) )
                {
                    list.removeElementAt(i);
                    return;
                }
            }
            else
            if(fromClass.equals(s.fromClass) && fromFnc.equals(s.fromFnc) &&
               name.equals(s.name) )
            {
                list.removeElementAt(i);
                return;
            }
        }
    }
        
    
    /* return values:
     * -1 - var not found
     *  0 - wrong type
     *  1 - type is ok.
     */
    public int typeCheck(String type, String fromClass, String fromFnc, String name)
    {
        Symbol s;
        // find element (class and name)
        for(int i = 0; i < list.size(); i++)
        {
            s = (Symbol) list.elementAt(i);
            if(fromClass.equals(s.fromClass) && name.equals(s.name) )
            {
                // check if variable is from the same function 
                if(fromFnc != null && !fromFnc.equals(s.fromFnc) )
                    continue;
                    
                // compare types
                if(type.equals(s.type) )
                    return(1);
                else
                    return(0);
            }   
        }
        return(-1);
    }
    

    // returns the type of a given variable
    public String varType(String fromClass, String fromFnc, String name)
    {
        Symbol s;
        // find element (class and name)
        for(int i = 0; i < list.size(); i++)
        {
            s = (Symbol) list.elementAt(i);
            if(fromClass.equals(s.fromClass) && name.equals(s.name) )
                return(s.type);
        }
        return(null);
    }

    // returns the size of function in bytes == 2 * #local variables
    public int fncSize(String className, String fncName)
    {
        Symbol s;
        int i, size = 0;

        // summerize all variables with same classname with no [tilh�rsforhold] to a function
        for(i = 0; i < list.size(); i++)
        {
            s = (Symbol) list.elementAt(i);
            if(className.equals(s.fromClass) && fncName.equals(s.fromFnc))
                size++;
        }

        // we *2 in indexsize, since all elements is implemented as 
        // 16 bits == 2 elements on the stack/hob
        return(size*2);
    }
    
    
    // returns size of class in bytes = 2 * #local variables
    public int classSize(String className)
    {
        Symbol s;
        int i, size = 0;

        // summerize all variables with same classname with no tilh�rsforhold to a function
        for(i = 0; i < list.size(); i++)
        {
            s = (Symbol) list.elementAt(i);
            if(className.equals(s.fromClass) && s.fromFnc == null)
                size++;
        }
        // we *2 in indexsize, since all elements is implemented as 
        // 16 bits == 2 elements on the stack/hob
        return(size*2);
    }
    
    
    public int varIndex(String fromClass, String fromFnc, String name)
    {
        Symbol s;
        // find element (class and name)
        for(int i = 0; i < list.size(); i++)
        {
            s = (Symbol) list.elementAt(i);
            if(fromClass.equals(s.fromClass) && name.equals(s.name) )
            {
                
                // check if variable is from the same function (if from a function)
                if(fromFnc == null && s.fromFnc == null)
                    return(s.index);
                else    
                if(fromFnc != null && fromFnc.equals(s.fromFnc) )
                    return(s.index);
            }   
        }

        // failed to find variable, 
        
        return(-1); // failed to find variable!
    }
    

/*
// for test only!
static void main(String[] a)
{
        SymbolTable s = new SymbolTable();
        
        s.add("int", "A", "f", "i");
        s.add("int", "A", null, "k");
        s.add("char", "A", "f", "j");

        s.add("int", "A", "g", "j");
        s.add("int", "B", null, "k");
        s.add("char", "A", null, "x");

        System.out.println("A = " + s.classSize("A") );
        System.out.println("A.f() = " + s.fFncSize("A", "f") );
        System.out.println("pos A.f.j = " + s.varIndex("A", "f", "j") );
        System.out.println("pos B.k = " + s.varIndex("B", null, "k") );
        System.out.println("A.g() = " + s.fncSize("A", "g") );

        if(s.typeCheck("int", "A", null, "i") == true)  System.out.println("i er int");
        else System.out.println("i != int ");

        if(s.typeCheck("char", "A", "f", "j") == true)  System.out.println("A.f().j er char");
        else System.out.println("i != int ");

        if(s.typeCheck("int", "A", "f", "j") == true)   System.out.print("A.f.j er int");
        else System.out.println("A.f.j != int ");

}
*/

}
\end{verbatim}
\end{footnotesize}
